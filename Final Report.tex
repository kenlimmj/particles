\documentclass{scrartcl}
\usepackage{dominatrix}
\usepackage{tikz}
\usepackage{pgfplots}
\pgfplotsset{
  every axis/.append style={font=\small},
  compat=newest
}

\begin{document}
  \begin{framed}
  CS 5220 Introduction to Parallel Programming \hfill Fall 2015 \\
  Kenneth Lim (\href{mailto:kl545@cornell.edu}{kl545}), Scott Wu (\href{mailto:ssw74@cornell.edu}{ssw74}), Robert Chiodi (\href{mailto:rmc298@cornell.edu}{rmc298}), Ravi Patel (\href{mailto:rgp62@cornell.edu}{rgp62})  \hfill Final Project \hspace{-3ex}
  \end{framed}
  
  
  \section{Introduction}
  
  \subsection{Smooth Particle Hydrodynamics (SPH)}
  
  \subsection{Code Bases}
  \subsubsection{C Code}
  
  
  \subsubsection{Fortran Code}
    
    
    
  \section{Profiling and Serial Optimization}  
  \subsection{C Code}

  \subsubsection{Profiling}
  
  \subsubsection{Optimizations}
  \subsubsection{Compiler Flags}
  
  \subsection{Fortran Code}
  \subsubsection{Profiling}
  
  \subsubsection{Optimizations}
  \subsubsection{Compiler Flags}
  
  
  \section{Parallelization}
  \subsection{C Code}  
  \subsubsection{OpenMP}
  \begin{table}
  	\begin{tabular}{| c | c | c | c | c | c | c | c |}
  		\hline
  	Case & \# Particles & LLC ($x$,$y$,$z$) & URC ($x$,$y$,$z$) & \# Threads & Time (s) & SS (\%)& SSE (\%) \\ \hline
	1 & 34,328 & (0.0, 0.0, 0.0) & (0.5, 0.5, 0.5) & 1 & & & \\ \hline		  		
	2 & 34,328 & (0.0, 0.0, 0.0) & (0.5, 0.5, 0.5) & 2 & & & \\ \hline		  		
	3 & 34,328 & (0.0, 0.0, 0.0) & (0.5, 0.5, 0.5) & 4 & & & \\ \hline		  		
	4 & 34,328 & (0.0, 0.0, 0.0) & (0.5, 0.5, 0.5) & 8 & & & \\ \hline		  		
	5 & 34,328 & (0.0, 0.0, 0.0) & (0.5, 0.5, 0.5) & 12 & & & \\ \hline		  		
	6 & 34,328 & (0.0, 0.0, 0.0) & (0.5, 0.5, 0.5) & 16 & & & \\ \hline		  		
	7 & 34,328 & (0.0, 0.0, 0.0) & (0.5, 0.5, 0.5) & 24 & & & \\ \hline		  		
	\end{tabular}
	\caption{Configuration of dam break simulations used for strong scaling study. LLC = Lower Left Corner, URC = Upper Right Corner, SS = Strong Scaling, SSE = Strong Scaling Efficiency.}
	\label{tab:ss}
  \end{table}
  
    \begin{table}
    	\begin{tabular}{| c | c | c | c | c | c | c | c |}
    		\hline
    		Case & \# Particles & LLC ($x$,$y$,$z$) & URC ($x$,$y$,$z$) & \# Threads & Time (s) & SS (\%)& SSE (\%) \\ \hline
    		1 & 2,197 & (0.0, 0.0, 0.0) & (0.5, 0.5, 0.5) & 1 & & & \\ \hline		  		
    		2 & 4,394 & (0.0, 0.0, 0.0) & (0.5, 0.5, 0.5) & 2 & & & \\ \hline		  		
    		3 & 8,788 & (0.0, 0.0, 0.0) & (0.5, 0.5, 0.5) & 4 & & & \\ \hline		  		
    		4 & 17,576 & (0.0, 0.0, 0.0) & (0.5, 0.5, 0.5) & 8 & & & \\ \hline		  		
    		5 & 26,364 & (0.0, 0.0, 0.0) & (0.5, 0.5, 0.5) & 12 & & & \\ \hline		  		
    		6 & 35,152 & (0.0, 0.0, 0.0) & (0.5, 0.5, 0.5) & 16 & & & \\ \hline		  		
    		7 & 52,728 & (0.0, 0.0, 0.0) & (0.5, 0.5, 0.5) & 24 & & & \\ \hline		  		
    	\end{tabular}
    	\caption{Configuration of dam break simulations used for weak scaling study. LLC = Lower Left Corner, URC = Upper Right Corner, SS = Strong Scaling, SSE = Strong Scaling Efficiency.}
    	\label{tab:ws}
    \end{table}
  
    \textbf{Strong Scaling:}
    
    \textbf{Weak Scaling:}
    
  \subsubsection{Cilk}
  \textbf{Strong Scaling:}
  
  \textbf{Weak Scaling:}
  \subsection{Fortran Code}    
  
  \subsubsection{OpenMP}
  \textbf{Strong Scaling:}
  
  \textbf{Weak Scaling:}  
  
  \subsubsection{MPI}
  \textbf{Strong Scaling:}
  
  \textbf{Weak Scaling:}

	
  \section{Summary and Future Work}

\end{document}
